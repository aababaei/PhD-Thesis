\documentclass[../thesis.tex]{subfiles}
\begin{document}
\begin{abstract}
\addcontentsline{toc}{section}{Abstract}
\noindent The abstract to be fixed later\dots

We investigate the dynamics of inertial particles in homogeneous isotropic turbulence, under one-way momentum coupling, using a new computational approach that incorporates the effect of long-range many-body aerodynamic interactions along with the short-range lubrication forces. The implementation couples Hybrid Direct Numerical Simulations (HDNS) with the analytical solutions of two rigid spheres moving in an unbounded fluid. Concerning the velocity field seen by the particles, the algorithm switches from the flow solution in terms of HDNS to analytical formulae when the separation distance between particles becomes comparable to their average radius. Standard HDNS is unable to correctly represent the short-range interactions since this method is based on the superposition of the Stokes solutions for single spheres. Our results show that for the turbulent kinetic energy dissipation rates typical of atmospheric clouds, the radial relative velocities (RRV) of the droplets increase, and the radial distribution function (RDF) decreases in the near-contact region if the lubricative forces are taken into account. These changes are more pronounced when the effect of gravity is considered. Away from the contact region, there is not much change in RRVs and RDFs. For turbulent clouds with lower dissipation rates lubrication forces significantly enhance the average RRV in the limit of low Stokes number. This enhancement, however, is statistically insignificant because the number of particle pairs at close proximity is very small. The effect of mass loading on the collision statistics is also investigated, demonstrating an increase in RRV and a reduction in RDF with the droplet concentration.

The collision efficiency of cloud droplets settling under gravity in quiescent air is investigated by means of numerical simulations. In the developed model, the droplets are represented either as rigid spheres or non-deformable liquid particles of finite viscosity. For the latter, both the internal circulation of the fluid and the mobility of interfaces are accounted for. The aerodynamic interaction, resulting from relative motion of the particles in a viscous medium, is evaluated by making use of a Stokes flow solution. The effect of non-continuum lubrication for the flow in the gap between surfaces of the droplets is also analyzed. This provides a more physical description of aerodynamic interactions valid for a wide range of the Knudsen number and gap sizes. In contrast to an earlier study by Rother \textit{et al}.\ (\textit{Int.\ J. Multiph.\ Flow} \textbf{146}, 103876, 2022), non-continuum lubrication and internal circulation effects have been analyzed separately. Additionally, rotational motion is considered for rigid particles. An objective comparison of the obtained results with the reference data has been performed as well. Compared to the standard continuum description of aerodynamic interaction for non-rotating rigid spherical particles, non-continuum lubrication and internal circulation effects both lead to a larger collision efficiency, whereas rotation reduces collision efficiency. In general, non-continuum lubrication has a larger impact on the collision efficiency compared to the internal circulation of drops which loses its influence as their inertia (size) increases. In numerical simulations, therefore, treating medium-sized cloud droplets as rigid particles is an accurate assumption, but considering non-continuum effects in their aerodynamic interaction is expected to alter the results. In the limit of a large viscosity ratio, the values of collision efficiency for the liquid drops and freely rotating rigid particles are in a quantitative agreement. Numerical aspects are also discussed, with a focus on assessing computational complexity and a new approach to algorithm parallelization. The simplified problem studied here is an important step towards improving the representation of aerodynamic interaction in systems with a large number of droplets interacting in a turbulent flow.

In this paper, the computational performance of a novel parallel code for simulating collision--coalescence of aerodynamically interacting droplets in turbulent flows is examined. Modeling such systems is essential for the quantitative description of processes relevant to precipitation formation. This knowledge, in turn, is crucial to develop more realistic parameterizations in numerical weather forecasting systems. The code is based on the standard Eulerian--Lagrangian approach. Direct numerical simulations (DNS) to solve the homogeneous isotropic turbulence are combined with analytical solutions of the Stokes flow to account for aerodynamic interaction (AI) among particles. Also, short-range interaction, the so-called lubrication forces, between particles is incorporated into the algorithm to improve the AI representation. The cubic computational domain is decomposed into smaller subdomains where calculations are handled by different processes. The Message Passing Interface (MPI) library is employed to transfer particle and flow data. This hybrid DNS (HDNS) algorithm enables tracking millions of interacting droplets in turbulent flows simulated on high-resolution meshes. The performance is evaluated by measuring the wall-clock time of major numerical operations. The results compare the time for treating AI, measured separately for long- and short-range forces, with the time required for the other particle operations as well as the time to advance the turbulent flow field. The effects of the number and size of the particles, the range of AI, and the number of processors are examined.

\vspace{6 pt}

\noindent\textbf{Keywords}: Cloud microphysics, Droplet collision statistics, Lubrication force, \linebreak 
Particle-laden turbulent flows, High-performance computing

\end{abstract}
\newpage
\end{document}
