\documentclass[../thesis.tex]{subfiles}

\begin{document}

\section*{Preface\label{sec:pre}}
\addcontentsline{toc}{section}{Preface}

This thesis has been compiled by merging the results of several studies on topics that were defined and addressed within the context of the \href{https://projekty.ncn.gov.pl/en/index.php?projekt_id=417540}{NCN project}. These studies have been published in the \href{https://doi.org/10.1017/jfm.2021.229}{\emph{Journal of Fluid Mechanics}}, the \href{https://doi.org/10.1016/j.ijmultiphaseflow.2021.103906}{\emph{International Journal of Multiphase Flow}}, and \href{https://doi.org/10.1103/PhysRevFluids.8.014102}{\emph{Physical Review Fluids}}. They were all aimed at improving the representation of aerodynamic interaction of cloud droplets via considering two-way momentum coupling as well as different short-range physical phenomena such as lubrication force, non-continuum flow in the gap between droplets, and the fluid droplet assumption. Section~\ref{sec:int} begins with a general introduction of cloud physics, the importance of accurately representing cloud microphysical processes, a survey of studies on the topic, and problems that this document plans to address. Section~\ref{sec:met} describes the mathematics of the problems at hand, governing equations, numerical tools and schemes, and implementation details of the codes utilized to carry out the simulations. Before presenting the results of numerical simulations, the influence of the key parameters that affect the collision statistics have been examined at the beginning of Section~\ref{sec:col}. Subsequently, the collision statistics under one-way momentum coupling considering lubrication effects have been presented, followed by a detailed analysis of the numerical performance of this implementation. Later, collision statistics of non-interacting droplets under two-way momentum coupling have been presented in this section. Afterwards, Section~\ref{sec:eff} focuses on further improvement of the representation of lubrication effects via taking non-continuum molecular effects and internal circulation of drops into account. In this section, the numerical scheme to compute gravitational collision efficiency of a pair of droplets interacting in a viscous fluid has been described. This is followed by the introduction and comparison of various force representations, and later presenting the collision efficiency by employing these representations. Every discussion on each set of results concludes with a summary that reviews the key points. Finally, Section~\ref{sec:con} concludes the thesis through a discussion on the important findings of these studies, and subsequently suggesting an outlook for the possible directions that can be taken in future studies.

Throughout this thesis, the two words ``particle'' and ``droplet'' have been interchangeably used in many instances except in Section \ref{sec:eff} where they indicate two separate models that either assume droplets as \emph{rigid particles} or \emph{fluid drops} with internal circulations (enclosed flows) that have mobile interfaces with the surrounding flow of air.

\newpage
\end{document}
