\documentclass[../thesis.tex]{subfiles}

\begin{document}

\section{Conclusions and outlook\label{sec:con}}

To be written

\paragraph{Outlook}

lubrication part:

The present results are a step forward in quantifying the effect of lubrication interaction on the collision statistics but are limited to the monodisperse systems and low mesh resolutions, or equivalently low Taylor-microscale flow Reynolds numbers. Therefore, potentially important direction of further research may be an extension of this analysis to polydisperse systems, covering a wide range of droplet sizes. In doing so, the need for random relocation after collisions is eliminated yielding more physically realistic results. Also, simulations at higher mesh resolutions may shed more light on the effect of aerodynamic interactions. It should be noted, however, that the numerical cost of simulations with larger domains is disproportionately much higher. This leap in complexity is mainly due to the need to track a large number of droplets necessary to maintain a desired mass loading. Also note that including lubrication interactions imposes additional constraints on the integration time step. Another direction possibly worthy of further investigation is the effect of lubrication when the continuum assumption of the fluid is no longer applicable. At such small separation distances the lubrication forces are lower than those predicted under the continuum fluid assumption.


%\bibliographystyle{bibstyle}
%\bibliography{references}
\newpage
\end{document}
