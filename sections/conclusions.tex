\documentclass[../thesis.tex]{subfiles}

\begin{document}

\section{Conclusions and outlook\label{sec:con}}
This thesis was aimed at quantifying the collision statistics of medium-sized (10--60~$\mu$m) cloud droplets interacting in turbulent air. Through the improvements on the already established models utilized here, the aerodynamic interaction of cloud droplets is better represented, making it possible to conduct simulations in conditions more similar to those in realistic clouds. This paves the way to the development of more accurate parameterizations for weather forecasting models. To perform the simulations, an Eulerian--Lagrangian approach was adopted that combines the direct numerical simulations of the background turbulent flow field with a Lagrangian particle tracking scheme. The momentum transfer between the flow and droplets have been modeled using two different schemes referred to as one-way and two-way momentum coupling.

The first set of results have been obtained under one-way momentum coupling, only allowing the droplets to feel the background flow (flow--droplet interaction) via viscous drag forces, but the reverse coupling (droplet--flow interaction) was ignored---an assumption applicable to dilute systems. Moreover, the aerodynamic interaction among the droplets (droplet--droplet AI) was taken into consideration together with an approach that accounts for short-range lubrication force between droplets that are in proximity. To do so, the standard HDNS method needed to be combined with the exact analytical formula for the lubrication forces. Merging these two approaches allows us to jointly represent the effect of many-body interaction among widely separated droplets as well as the effect of lubrication forces between closely separated pairs of droplets. The impacts of the inertia and mass loading of water droplets in the absence and presence of gravity on statistics were additionally investigated. However, the results were limited to monodisperse systems simulated only on a grid of resolution $64^3$, limiting the analysis to a low Taylor-microscale flow Reynolds number $R_{\lambda}=76$. Based on the results, the RRV is slightly enhanced when AI is taken into consideration, particularly in the systems with droplets of low inertia. On the other hand, in the absence of gravity the AI reduces the RDF such that the reduction depends on the droplet inertia. Another observation showed that lubrication effect intensifies the influence of AI on statistics. Additionally, the effect of lubrication forces is more pronounced in systems with larger energy dissipation rates (higher inertia), especially if gravitational settling is considered. This is based on the comparison of standard HDNS with simulations performed using the new approach that has a full representation of AI, i.e.\ HDNS with lubrication. Once the mass loading grows, the kinematic collision statistics reveal an increase in the RRV with the droplet number density, while the RDF monotonically decreases. This stems from the combined effect of many-body AI, gravity and the post-collision treatment model through which one of the droplets is relocated after collision.

The next part of thesis was focused on examining the computational performance of the implementation for HDNS with lubrication. The performance was assessed based on a number of testing simulations by measuring the time required to conduct individual operations. The focus was on computation time for aerodynamic interaction and lubrication forces, and the results were directly compared with the time required for the other tasks related to droplet tracking and computing collision statistics as well as the time to advance the turbulent flow field. The factors examined were the number and size of the particles in the domain, the size of the region in which lubrication effects are considered, and the number of processors to simulate three different systems. The first three factors increased computation time. Moreover, it was observed that the scalability of the code improves with increasing resolution of the computational grid. 

In the following part of the discussion, several simulations were performed under two-way momentum coupling while neglecting droplet--droplet AI. The reverse momentum coupling that represents particle--flow interaction is accounted for through an additional source term in the equations for the conservation of momentum in the flow field. To address the effect of the range of turbulent scales on the dispersed phase, the simulations were performed on meshes of two different resolutions: $64^3$ and $256^3$. Since the simulations with a large mass loading on meshes of higher resolutions are prohibitively expensive, we had to use a parameterization, the so-called super-droplet approach, that significantly reduces computational cost but impairs accuracy. The results demonstrated that for non-settling droplets, the monodisperse RDF decreases with the mass loading in all modeled cases, i.e.\ at different droplet inertia and Reynolds numbers. In other words, the spatial distribution of the particles becomes more uniform when the particle concentration increases. This effect is a consequence of damping of the vortical structures by moving particles. In the presence of gravity, the RDF is non-monotonic. Compared to simulations under one-way momentum coupling, in the range of low mass loadings ($\Phi_m$) the RDF increases slightly with $\Phi_m$. This enhancement results from the interaction of the droplets with the vortices forced by the motion of the dispersed phase. At higher mass loadings, the RDF systematically decreases thanks to the stronger swirling strength that makes the system more uniform and homogeneous. We also noticed that the results are sensitive to the super-droplet parameterization, particularly to the $M$ parameter. Thus, the results obtained using a larger $M$, computed in simulations at higher $R_{\lambda}$, can only be interpreted qualitatively. Furthermore, we observed a significant enhancement of the RRV by increasing mass loading. This is due to the momentum transfer from droplets to the fluid, which causes more effective decorrelation of their motion. Results from simulations on larger meshes reveal a huge increase in RRV at higher mass loadings. Overall, the RRV is more sensitive to the $M$ parameter than the RDF. 

The rest of the thesis was focused on improvements in short-range lubrication force between interacting droplet via considering two physically more accurate models for interacting pairs, namely non-continuum lubrication and fluid drop. The former representation is based on the assumption of a non-continuum flow in the gap between droplets when the size of the gap between them is comparable to the mean free path of air. Also, under the fluid drop model, instead of a spherical rigid particle assumption, the droplets are treated as fluids with internal circulations that have mobile interfaces. Each of these phenomena modifies the lubrication force representation compared to a spherical rigid particle assumption with a continuous flow in the gap. The results quantify the collision efficiency of two non-deformably spherical particles settling under gravity in quiescent air. Most of the simulations were performed using settings characteristic for typical cloud droplets of radii in the range 0.5 to 30~$\mu$m. The results showed that, thanks to a lower drag, both effects lead to an increase in collision efficiency with non-continuum lubrication exhibiting a much larger enhancement compared to the fluid drop model for AI. Moreover, at higher inertia of larger droplet pairs, the importance of fluid drop assumption pales in comparison with inertial effects, while the non-continuum lubrication maintains its importance by a noticeable increase in collision efficiency. In numerical simulations, therefore, treating medium-sized cloud droplets as rigid particles is a reasonable assumption, but considering non-continuum effects in their aerodynamic interaction is expected to affect the results. 

\paragraph{Outlook}

The results presented in this thesis are a step forward in quantifying the effect of aerodynamic interaction, whether by including lubrication or a two-way momentum coupling scheme, on the collision statistics of cloud droplets. An accurate description of this process finds application in numerical weather prediction systems. The results of investigation on the effects of lubrication interaction were limited to monodisperse systems and low mesh resolutions, or equivalently low Taylor-microscale flow Reynolds numbers. Therefore, potentially important direction of further research may be an extension of this analysis to polydisperse systems, covering a wide range of droplet sizes. In doing so, the need for random relocation after collisions is eliminated yielding physically more realistic results. Furthermore, it has been observed that the scalability of the code improves with increasing resolution of the computational grid. This is an encouraging perspective to run simulations on even larger computational meshes, or equivalently larger Reynolds numbers. Also, simulations at higher mesh resolutions may shed more light on the effect of aerodynamic interactions. It should be noted, however, that the numerical cost of simulations with larger domains is disproportionately much higher. This leap in complexity is mainly due to the need to track a large number of droplets necessary to maintain a desired mass loading. Also note that including lubrication interactions imposes additional constraints on the integration time step. Our analysis of collision statistics under two-way momentum coupling was rather limited here. Further work is warranted to more thoroughly explore the parameter space, i.e.\ the Reynolds number and mass loading. Also, regarding the numerical method, working out a remedy against the adverse effect of the super-droplet parameterization on the accuracy of collision statistics. Another direction worthy of further investigation is the effect of lubrication when either the non-continuum molecular effects or a fluid drop model is adopted to represent AI. At small separation distances the lubrication forces are lower under each of these representations. The last part of our discussion was aimed at quantifying these effects in a simpler problem, namely gravitational collision efficiency of a pair of cloud droplets settling in still air. The impact of each of these effects on collision statistics of aerodynamically interacting cloud droplets in turbulence remains to be addressed. Finally, a thorough investigation that examines how each of these AI models (two-way momentum coupling, non-continuum lubrication, etc.) affects the average droplet settling velocity and fluctuations in droplet velocities as well as accelerations would indeed be of interest.

%\bibliographystyle{bibstyle}
%\bibliography{references}
\newpage
\end{document}
